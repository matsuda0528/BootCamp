\documentclass[12pt]{jsarticle}
\usepackage[dvipdfmx]{graphicx}
\textheight = 25truecm
\textwidth = 18truecm
\topmargin = -1.5truecm
\oddsidemargin = -1truecm
\evensidemargin = -1truecm
\marginparwidth = -1truecm

\def\theenumii{\Alph{enumii}}
\def\theenumiii{\alph{enumiii}}
\def\labelenumi{(\theenumi)}
\def\labelenumiii{(\theenumiii)}
\def\theenumiv{\roman{enumiv}}
\def\labelenumiv{(\theenumiv)}
\usepackage{comment}
\usepackage{url}

%%%%%%%%%%%%%%%%%%%%%%%%%%%%%%%%%%%%%%%%%%%%%%%%%%%%%%%%%%%%%%%%
%% sty/ にある研究室独自のスタイルファイル
\usepackage{jtygm}  % フォントに関する余計な警告を消す
\usepackage{nutils} % insertfigure, figref, tabref マクロ

\def\figdir{./figs} % 図のディレクトリ
\def\figext{pdf}    % 図のファイルの拡張子

\begin{document}
%%%%%%%%%%%%%%%%%%%%%%%%%%%%
%% 表題
%%%%%%%%%%%%%%%%%%%%%%%%%%%%
\begin{center}
{\LARGE SlackBotプログラムの報告書}
\end{center}

\begin{flushright}
  2020/4/28\\
  松田 陸斗
\end{flushright}
%%%%%%%%%%%%%%%%%%%%%%%%%%%%
%% 概要
%%%%%%%%%%%%%%%%%%%%%%%%%%%%
\section{はじめに}
\label{sec:introduction}
本資料は,B4新人研修のRubyによるSlackBotプログラムの作成の報告書である.
本資料では,SlackBotプログラムの作成に関して,理解できなかった部分,作成できなかった機能,自主的に作成した機能を述べる.

\section{課題内容}
以下の2つの機能をもつSlackBotプログラムをRubyで作成する.
\begin{enumerate}
\item 任意の文字列を発言するプログラムの作成\\
受信した発言の中に''「hello」と言って''という文字列があった場合は,''hello''と発言する
\item SlackBotプログラムへの機能追加\\
Slack以外のWebサービスのAPIやWebhookを利用した機能を追加する.
\end{enumerate}

\section{理解できなかった部分}
\begin{enumerate}
\item 	localからSlack.comにポストリクエストを送るとエラーが出る\\
\end{enumerate}

\section{作成できなかった機能}
\begin{enumerate}
\item 時報機能の作成\\
当初の予定では,cronを動かすことで時報機能を作成しようと考えていた.
しかし,Herokuの無料サーバでは30分アクセスがないとシャットダウンしてしまうため,有料サーバを使う,無料サーバに外部から30分おきにアクセスをするなどの工夫が必要である.
今回作成する課題は,WebサービスのAPIやWebhookを利用した機能の追加のため,時報機能の実装は見送った.
\end{enumerate}

\section{自主的に作成した機能}
\begin{enumerate}
\item 天気を取得する機能\\
''@matsudabot (場所)の天気''という発言を受信すると,(場所)の天気を取得する.
\item ニュースを取得する機能\\
''@matsudabot ニュース''という発言を受信すると,トップニュースから$1$件取得する.
また,''@matsudabot ''(検索ワード)''のニュース''という発言を受信すると,(検索ワード)を含むニュースを$1$件取得する.
さらに,''@matsudabot ニュースを3件''という発言のように,件数を指定すると,指定した件数のニュースを取得する.
\item クイズを出題する機能\\
''@matsudabot クイズ''という発言を受信すると,クイズをランダムに取得する.
そして,次の発言をクイズの解答として受信する.
\end{enumerate}

\bibliographystyle{ipsjunsrt}
\bibliography{mybibdata}

\end{document}
