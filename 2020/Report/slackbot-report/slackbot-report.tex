\documentclass[12pt]{jsarticle}
\usepackage[dvipdfmx]{graphicx}
\textheight = 25truecm
\textwidth = 18truecm
\topmargin = -1.5truecm
\oddsidemargin = -1truecm
\evensidemargin = -1truecm
\marginparwidth = -1truecm

\def\theenumii{\Alph{enumii}}
\def\theenumiii{\alph{enumiii}}
\def\labelenumi{(\theenumi)}
\def\labelenumiii{(\theenumiii)}
\def\theenumiv{\roman{enumiv}}
\def\labelenumiv{(\theenumiv)}
\usepackage{comment}
\usepackage{url}

%%%%%%%%%%%%%%%%%%%%%%%%%%%%%%%%%%%%%%%%%%%%%%%%%%%%%%%%%%%%%%%%
%% sty/ にある研究室独自のスタイルファイル
\usepackage{jtygm}  % フォントに関する余計な警告を消す
\usepackage{nutils} % insertfigure, figref, tabref マクロ

\def\figdir{./figs} % 図のディレクトリ
\def\figext{pdf}    % 図のファイルの拡張子

\begin{document}
%%%%%%%%%%%%%%%%%%%%%%%%%%%%
%% 表題
%%%%%%%%%%%%%%%%%%%%%%%%%%%%
\begin{center}
{\LARGE SlackBotプログラムの報告書}
\end{center}

\begin{flushright}
  2018/4/28\\
  松田 陸斗
\end{flushright}
%%%%%%%%%%%%%%%%%%%%%%%%%%%%
%% 概要
%%%%%%%%%%%%%%%%%%%%%%%%%%%%
\section{はじめに}
\label{sec:introduction}
本資料は,B4新人研修のRubyによるSlackBotプログラムの作成の報告書である.
本資料では,SlackBotプログラムの作成に関して,理解できなかった部分,作成できなかった機能,自主的に作成した機能を述べる.

\section{理解できなかった部分}
\begin{enumerate}
\item 	localからSlack.comにポストリクエストを送るとエラーが出る\\
\end{enumerate}

\section{作成できなかった機能}
\begin{enumerate}
\item 認証を必要とするAPIの利用\\
OAuth認証の設定がよくわからなかった.
\item 時報機能\\
herokuの無料プランのサーバでは,30分アクセスがないとシャットダウンしてしまうため,cronなどが使えない.
\end{enumerate}

\section{自主的に作成した機能}
\begin{enumerate}
\item 天気を取得する機能\\
任意の場所の天気を取得する.場所の名前には,予め登録されてある名前を指定する必要がある.トリガーとなる言葉は"<場所>の天気"である.
\item ニュースを取得する機能\\
最新のニュースを任意数取得する.検索したい言葉をダブルクォーテーションで囲むと,この言葉に関するニュースを取得する.トリガーとなる言葉は,"ニュース"である.
\item クイズを出題する機能\\
コンピュータに関するクイズをランダムに出題する.トリガーとなる言葉は"クイズ"である.
\end{enumerate}

\section{おわりに}
\label{sec:conclusion}
本資料では打ち合わせ資料のテンプレートを示した.
また,図表の挿入例や参考文献の例を挙げた.
今後は,このテンプレートを基に資料を作成する.

\bibliographystyle{ipsjunsrt}
\bibliography{mybibdata}

\end{document}
