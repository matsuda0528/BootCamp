\documentclass[12pt]{jsarticle}
\usepackage[dvipdfmx]{graphicx}
\textheight = 25truecm
\textwidth = 18truecm
\topmargin = -1.5truecm
\oddsidemargin = -1truecm
\evensidemargin = -1truecm
\marginparwidth = -1truecm

\def\theenumii{\Alph{enumii}}
\def\theenumiii{\alph{enumiii}}
\def\labelenumi{(\theenumi)}
\def\labelenumiii{(\theenumiii)}
\def\theenumiv{\roman{enumiv}}
\def\labelenumiv{(\theenumiv)}
\usepackage{comment}
\usepackage{url}

%%%%%%%%%%%%%%%%%%%%%%%%%%%%%%%%%%%%%%%%%%%%%%%%%%%%%%%%%%%%%%%%
%% sty/ にある研究室独自のスタイルファイル
\usepackage{jtygm}  % フォントに関する余計な警告を消す
\usepackage{nutils} % insertfigure, figref, tabref マクロ

\def\figdir{./figs} % 図のディレクトリ
\def\figext{pdf}    % 図のファイルの拡張子

\begin{document}
%%%%%%%%%%%%%%%%%%%%%%%%%%%%
%% 表題
%%%%%%%%%%%%%%%%%%%%%%%%%%%%
\begin{center}
{\LARGE 資料タイトル}
\end{center}

\begin{flushright}
  2018/4/1\\
  小倉 伊織
\end{flushright}
%%%%%%%%%%%%%%%%%%%%%%%%%%%%
%% 概要
%%%%%%%%%%%%%%%%%%%%%%%%%%%%
\section{はじめに}
\label{sec:introduction}
本資料は打ち合わせ資料のテンプレートを示した資料である.
本資料を作成するにあたって,学士卒業論文テンプレートを参考にした.
はじめにでは,本資料の概要や背景を説明する.
2章に箇条書きの例,図の挿入の例,表の例,および参考文献の例について記載している.


\section{章}\label{sec:hoge}
\subsection{節}\label{subsec:hoge}
\subsubsection{項}\label{subsubsec:hoge}

章,節,および項の適切な名前を考える.

\subsection{箇条書きの例}
箇条書きを用いて分かりやすく表現する.
\begin{enumerate}
\item 項目1
\item 項目2
  \begin{enumerate}
  \item 項目A
  \item 項目B
  \end{enumerate}
\item 項目3
\end{enumerate}

(1)や(2)を別の文字に変えたい場合は,descriptionを使用する.
\begin{description}
\item[(問題1)] (問題1)が発生
\item[(問題2)] (問題2)が発生
\end{description}

\subsection{図の挿入例}
図を挿入する際は挿入する図をpdfに変換し,figsフォルダに入れる.
また,使用する図のページに合わせて,project.mkのFIG\_PAGESを変更する.
挿入した図を\figref{fig1}に示す.
図に対する説明を記載する.

% Usage: \insertfigure[magnification]{label}{filename}{caption}{ecaption}
% Exsample: \insertfigure[0.8]{lbl:fig3}{fig3}{日本語キャプション}{English Caption}
% default magnification is 0.9.
\insertfigure[0.8]{fig1}{fig1}{よくわかる図その1}{Awesome Figure 1.}

\subsection{表の例}
表を入れる際は過度に罫線を入れすぎないように注意する.
表の例を\tabref{tab:time_range_ratio}に示す.
表に対する説明を記載する.

\begin{table}[tb]
  \begin{center}
    \caption{作業時間の発生頻度}\label{tab:time_range_ratio}
    \ecaption{Frequency of task ocurrences.}
    \begin{tabular}{r|r|r|r}
      \hline\hline
      \multicolumn{1}{l|}{通番} & \multicolumn{1}{l|}{作業時間 (分)} & \multicolumn{1}{l|}{発生回数 (回)} & \multicolumn{1}{l}{累積割合 (\%)}\\
      \hline
      1 & {\texttt{120}} $\sim$ {\texttt{150}} & 17 & 40\\
      2 & {\texttt{150}} $\sim$ {\texttt{180}} & 12 & 67\\
      3 & {\texttt{90}} $\sim$ {\texttt{120}} & 7 & 84\\
      4 & {\texttt{180}} $\sim$ {\texttt{210}} & 4 & 93\\
      5 & $\sim$ {\texttt{90}} & 2 & 98\\
      6 & {\texttt{210}} $\sim$ & 1 & 100\\
      \hline
    \end{tabular}
  \end{center}
\end{table}

\subsection{参考文献の挿入例}
参考文献を記載する際はbibtexを利用する.
mybibdate.bibに参考文献の情報を記載する.
たとえば,乃村先生の論文\cite{nom2011c}を参考文献として記載する.

\section{おわりに}
\label{sec:conclusion}
本資料では打ち合わせ資料のテンプレートを示した.
また,図表の挿入例や参考文献の例を挙げた.
今後は,このテンプレートを基に資料を作成する.

\bibliographystyle{ipsjunsrt}
\bibliography{mybibdata}

\end{document}
