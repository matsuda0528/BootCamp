\documentclass[12pt]{jsarticle}
\usepackage[dvipdfmx]{graphicx}
\textheight = 25truecm
\textwidth = 18truecm
\topmargin = -1.5truecm
\oddsidemargin = -1truecm
\evensidemargin = -1truecm
\marginparwidth = -1truecm

\def\theenumii{\Alph{enumii}}
\def\theenumiii{\alph{enumiii}}
\def\labelenumi{(\theenumi)}
\def\labelenumiii{(\theenumiii)}
\def\theenumiv{\roman{enumiv}}
\def\labelenumiv{(\theenumiv)}
\usepackage{comment}
\usepackage{url}

%%%%%%%%%%%%%%%%%%%%%%%%%%%%%%%%%%%%%%%%%%%%%%%%%%%%%%%%%%%%%%%%
%% sty/ にある研究室独自のスタイルファイル
\usepackage{jtygm}  % フォントに関する余計な警告を消す
\usepackage{nutils} % insertfigure, figref, tabref マクロ

\def\figdir{./figs} % 図のディレクトリ
\def\figext{pdf}    % 図のファイルの拡張子

\begin{document}
%%%%%%%%%%%%%%%%%%%%%%%%%%%%
%% 表題
%%%%%%%%%%%%%%%%%%%%%%%%%%%%
\begin{center}
{\LARGE SlackBotプログラムの仕様書}
\end{center}

\begin{flushright}
  2020/4/28\\
  松田 陸斗
\end{flushright}
%%%%%%%%%%%%%%%%%%%%%%%%%%%%
%% 概要
%%%%%%%%%%%%%%%%%%%%%%%%%%%%
\section{はじめに}
\label{sec:introduction}
本資料は打ち合わせ資料のテンプレートを示した資料である.
本資料を作成するにあたって,学士卒業論文テンプレートを参考にした.
はじめにでは,本資料の概要や背景を説明する.
2章に箇条書きの例,図の挿入の例,表の例,および参考文献の例について記載している.

\section{概要}
本研修で作成したSlackBotは,Slackで"@matsudabot"から始まるチャットに反応し,続く文字列によって実装した機能を呼び出すものである.
本研修で作成したSlackBotは以下の機能を持つ.
\begin{enumerate}
\item 天気を取得し,表示する機能
\item ニュースを取得し,表示する機能
\item クイズを出題する機能
\end{enumerate}


\section{機能}
\subsection{天気機能}
天気を取得するために,Weather HacksというAPIを利用した.
Weather HacksはURLのパラメータに地域別に定義されたIDを指定する.
例に,久留米の天気を取得するURLを以下に示す.
\begin{verbatim}
http://weather.livedoor.com/forecast/webservice/json/v1?city=400040
\end{verbatim}
実装では,地域名とIDの対応表を作成し,地域名を入力から受け取ることができる仕様にしている.
\subsection{ニュース機能}
ニュースを取得するために,NewsAPIを利用した.
NewsAPIで提供されているAPIには,トップニュースを取得するためのAPIと,すべてのニュースを取得するAPIの二種類がある.
実装では,検索ワードを指定した場合には,すべてのニュースから検索し,検索ワードの指定がない場合には,トップニュースからニュースを取得している.
また,表示するニュースの件数を指定することができる.

\subsection{クイズ機能}
クイズを取得するために,OPEN TRIVIA DATABASEというAPIを利用した.
OPEN TRIVIA DATABASEはデータベースからクイズをランダムに取得できるAPIである.

\section{動作環境}
\begin{table}[h]
\begin{center}
\caption{動作環境}\label{tab:2}
\ecaption{Operationg environment.}
\begin{tabular}{l|l}
\hline\hline
\multicolumn{1}{l|}{項目} & \multicolumn{1}{l}{内容}\\
\hline
OS & Debian 10\\
CPU & Intel(R) Core(TM) m3-6Y30 CPU @ 0.90GHz 1.51GHz\\
メモリ & 512MB\\
Ruby & ruby 2.5.5p157\\
Ruby Gem & bundler 2.1.4\\
& termann 1.0.2\\
& rack 2.0.4\\
& rack-protection 2.0.1\\
& sinatra 2.0.1\\ 
& tilt 2.0.8\\
\hline
\end{tabular}
\end{center}
\end{table}

\section{動作確認済み環境}
\section{使用方法}

\subsection{天気機能}
天気を取得するための最も簡素なメッセージは以下である.
\begin{center}
@matsudabot <地域名>の天気\\
\end{center}
"<地域名>の天気"の後に文字があっても正常に呼び出される.
実際に想定されるメッセージの例を以下に示す.
\begin{center}
@matsudabot 岡山の天気は?\\
@matsudabot 神戸の天気を教えて\\
\end{center}
ただし,"<地域名>の天気"の前に文字を入れてはならない.
\subsection{ニュース機能}
ニュースを取得する機能は以下の2つに分けられる.
\begin{enumerate}
\item トップニュースからニュースを取得する機能.
\item すべてのニュースから検索ワードを含むニュースを取得する機能.
\end{enumerate}
\subsubsection{トップニュースから取得}
検索ワードを指定しない場合,トップニュースからニュースを取得する.
ニュースを取得するための最も簡素なメッセージは以下である.
\begin{center}
@matsudabot ニュース
\end{center}
\subsubsection{検索ワードを含むニュースを取得}
検索ワードはダブルクォーテーションでしていする.
検索ワードを含むニュースを取得する最も簡素なメッセージは以下である.
\begin{center}
@matsudabot "<検索ワード>"ニュース
\end{center}
\subsection{クイズ機能}
クイズを取得するための最も簡素なメッセージは以下である.
\begin{center}
@matsudabot クイズ
\end{center}

\section{エラー処理と保証しない動作}
保証しない動作を以下に示す.
\begin{enumerate}
\item ニュースの検索ワードに"件"が入っている場合.\\
\item ニュースの検索ワードに数字が入っていて,表示件数をしていする場合.\\
\item クイズの回答に30分以上かかる場合.\\
\end{enumerate}
\section{おわりに}
\label{sec:conclusion}
本資料では打ち合わせ資料のテンプレートを示した.
また,図表の挿入例や参考文献の例を挙げた.
今後は,このテンプレートを基に資料を作成する.

\bibliographystyle{ipsjunsrt}
\bibliography{mybibdata}

\end{document}
